\documentclass[12pt,fullpage]{article}
\usepackage{latexsym}
\oddsidemargin=-0.75cm
\evensidemargin=-0.75cm
\textwidth=17.60cm
\textheight=23cm
\topmargin=-1.50cm
\headsep=1cm

\title{\LARGE\bf BU CAS CS 320: Concepts of Programming Languages\break\break
Midterm-1 Examination}
\author{Instructor: Hongwei Xi}
\date{\today}
%
%\date{Time: 1:00-3:00 \kern12pt Room: SHA 2?? \break Date: the 11th of June, 2024}
%
\newtheorem{lemma}{Lemma}
\newtheorem{question}[lemma]{Question}
\def\fillsquare{\kern2pt\raise0.25pt
    \hbox{$\vcenter{\hrule height0pt \hbox{\vrule width5pt height5pt} \hrule height0pt}$}}
\newenvironment{solution}{%
\vskip6pt{\em Solution\kern6pt}}{%
{\unskip\nobreak\hfill\penalty50\kern4pt\hbox{}\nobreak\hfill\fillsquare}\vskip6pt}

\begin{document}
\maketitle

\def\exp#1{\mbox{\tt #1}}
\def\val#1{\mbox{\it #1}}
\def\mark{\mbox{$\star$}}

\noindent
Name:\underline{\hbox to 144pt{\hss}}
\hfill
Score:\underline{\hbox to 144pt{\hss}}

\vspace{12pt}
\begin{center}
\Large
\begin{tabular}{l|r|r|p{36pt}}
No. & Points & Answer & Score \\ \hline \hline
%%
01. & 10 & & \\ \hline
02. & 10 & & \\ \hline
03. & 10 & & \\ \hline
04. & 10 & & \\ \hline
05. & 10 & & \\ \hline
06. & 10 & & \\ \hline
07. & 20 & & \\ \hline
08. & 20 & & \\ \hline

%%
Total & 80 & & \\ \hline \hline

%%
\end{tabular} \\[24pt]
\end{center}
\vfill\newpage

\begin{center}
{\huge\bf Please, no computers are allowed!} \\[12pt]
\end{center}
\noindent
You can choose to solve Q5+Q6 or just solve Q7. If
you give a solution to Q7, then your solutions to Q5 and Q6
(if given) are not graded.

\begin{question}~~
\begin{verbatim}
HX-2024-06-11: 10 points
Fibonacci numbers can be recursively
defined as follows:
//
fun fibonacci(x: int): int =
  if x >= 2
  then fibonacci(x-2)+fibonacci(x-1) else x
//
Please give a direct non-combinator-based
tail-recursive implementation of fibonacci:
//
fun fibonacci_trec(x: int): int = ...
//
PLEASE NOTE THAT YOU CANNOT MAKE NON-TAIL-
RECURSIVE CALLS in your implementation. If you
do, your implementation is DISQUALIFIED.
\end{verbatim}
\end{question}

%%%%%%%%%%%%%%%%%%%%%%%%%%%%%%%%%%%%%%%%%%%%%%%%%%%%%%%%%%%%%%%%%%%%%%%%
\vfill\newpage
%%%%%%%%%%%%%%%%%%%%%%%%%%%%%%%%%%%%%%%%%%%%%%%%%%%%%%%%%%%%%%%%%%%%%%%%

\begin{question}~~
\begin{verbatim}  
HX-2024-06-11: 10 points
Fibonacci numbers can be recursively
defined as follows:
//
fun fibonacci(x: int): int =
  if x >= 2
  then fibonacci(x-2)+fibonacci(x-1) else x
//
Please use int1_foldright to implement the
function fibonacci WITHOUT using recursion.
PLEASE NOTE THAT YOU CANNOT DEFINE RECURSIVE
FUNCTIONS IN YOUR IMPLEMENTATION. If you do,
your implementation is DISQUALIFIED.
(*
Please use int1_foldright (not int1_foldleft)
*)
fun
fibonacci_nrec(x: int): int = // write your code below
\end{verbatim}
\end{question}

%%%%%%%%%%%%%%%%%%%%%%%%%%%%%%%%%%%%%%%%%%%%%%%%%%%%%%%%%%%%%%%%%%%%%%%%
\vfill\newpage
%%%%%%%%%%%%%%%%%%%%%%%%%%%%%%%%%%%%%%%%%%%%%%%%%%%%%%%%%%%%%%%%%%%%%%%%

\begin{question}~~
\begin{verbatim}  
############################################################

fun f91(x) = if x > 100 then x - 100 else f91(f91(x+11))

1. How many recursive calls are in the body of the above function? (1 point)

2. How many tail-recursive calls are in the body of the above function? (1 point)

3. Is the above implementation of f91 tail-recursive? (1 point)

############################################################

fun ghaap(m: int, res: int) : int =
  if m >= 0 then ghaap (m-1, m*res) else res

4. What is the value of ghaap(5, 5)? (2 points)

############################################################

Let us define the following function foo:

fun foo(x: int): int = (if x >= 0 then foo(2*x+1) else x) * x

5. What value is returned after the evaluation of foo(0)? (2 points)

############################################################

val global: int = 0
val tricky = let
  fun f(i: int) : int =
  let
    val global = global + i
  in
    if i < 10 then f(i+1) else global
  end
in f(0) end

6. What is the value of tricky? (3 points)

############################################################
\end{verbatim}
\end{question}

%%%%%%%%%%%%%%%%%%%%%%%%%%%%%%%%%%%%%%%%%%%%%%%%%%%%%%%%%%%%%%%%%%%%%%%%
\vfill\newpage
%%%%%%%%%%%%%%%%%%%%%%%%%%%%%%%%%%%%%%%%%%%%%%%%%%%%%%%%%%%%%%%%%%%%%%%%

\begin{question}~~
\begin{verbatim}
(*
//
// HX-2024-06-11: 10 points
//
Recall the iforall combinator:
//
type
('seq, 'elt) iforall =
'seq * (int * 'elt -> bool) -> bool
//
For instance, list_iforall can be implemented
as follows:
//
fun
list_iforall(xs, itest) =
let
  fun loop(i0, xs) =
    case xs of
      [] => true
    | x1 :: xs =>
      if itest(i0, x1) then loop(i0+1, xs) else false
in
  loop(0, xs)
end
//
(* ****** ****** *)
//
Given a sequence xs and an integer, lengte(xs, n)
returns true if and only if the length of xs >= n
Please give an implementation of lengte based on iforall:
(*
fun iforall_to_lengte(iforall) = fn(xs, n) => ...
*)
\end{verbatim}
\end{question}

%%%%%%%%%%%%%%%%%%%%%%%%%%%%%%%%%%%%%%%%%%%%%%%%%%%%%%%%%%%%%%%%%%%%%%%%
\vfill\newpage
%%%%%%%%%%%%%%%%%%%%%%%%%%%%%%%%%%%%%%%%%%%%%%%%%%%%%%%%%%%%%%%%%%%%%%%%

\begin{question}~~
\begin{verbatim}
//
// HX-2024-06-11: 10 points
//
Given a list of integers xs and an integer,
please implement a function that checks if
there exists a sublist of xs whose sum equals
the given integer:

fun
sublist_sum_test1(xs: int list, sum: int): bool

NOTE THAT INTEGERS CAN BE NEGATIVE. ALSO THE
ELEMENTS IN THE SUBLIST DO HAVE TO BE CONSECUTIVE.
                        ^^

You can use unrestricted recursion to solve this problem.
\end{verbatim}
\end{question}

%%%%%%%%%%%%%%%%%%%%%%%%%%%%%%%%%%%%%%%%%%%%%%%%%%%%%%%%%%%%%%%%%%%%%%%%
\vfill\newpage
%%%%%%%%%%%%%%%%%%%%%%%%%%%%%%%%%%%%%%%%%%%%%%%%%%%%%%%%%%%%%%%%%%%%%%%%

\begin{question}~~
\begin{verbatim}
//
// HX-2024-06-11: 10 points
//
Given a list of integers xs and an integer,
please implement a function that checks if
there exists a sublist of xs whose sum equals
the given integer:

fun
sublist_sum_test2(xs: int list, sum: int): bool

NOTE THAT INTEGERS CAN BE NEGATIVE. ALSO THE
ELEMENTS IN THE SUBLIST DO NOT HAVE TO BE CONSECUTIVE.
                        ^^^^^^

You can use unrestricted recursion to solve this problem.
\end{verbatim}
\end{question}

%%%%%%%%%%%%%%%%%%%%%%%%%%%%%%%%%%%%%%%%%%%%%%%%%%%%%%%%%%%%%%%%%%%%%%%%
\vfill\newpage
%%%%%%%%%%%%%%%%%%%%%%%%%%%%%%%%%%%%%%%%%%%%%%%%%%%%%%%%%%%%%%%%%%%%%%%%

\begin{question}~~
\begin{verbatim}
//
// HX-2024-06-11: 20 points
//
Given a list of integers xs,
please implement a function that computes
the maximum sum of a consecutive sublist of xs:

fun
sublist_max_sum(xs: int list): int

Yes, the integers in xs can be negative.

You can use unrestricted recursion to solve this problem.
\end{verbatim}
\end{question}

%%%%%%%%%%%%%%%%%%%%%%%%%%%%%%%%%%%%%%%%%%%%%%%%%%%%%%%%%%%%%%%%%%%%%%%%
\vfill\newpage
%%%%%%%%%%%%%%%%%%%%%%%%%%%%%%%%%%%%%%%%%%%%%%%%%%%%%%%%%%%%%%%%%%%%%%%%

\begin{question}~~
\begin{verbatim}
//
// HX-2024-06-11: 20 points
//
A sequence xs of integers captures '231'
if there are three integers a, b, and c
appearing as a subsequence [a,b,c] of xs
satisfying c < a < b. NOTE that a, b, and
c do not have to appear consecutively in xs.
//
For instance, [1,3,4,2] does capture '231'
For instance, [1,2,4,3] does not capture '231'
For instance, [1,2,3,4] does not capture '231'
(*
fun
perm_capture_231(xs: int list): bool = ...
*)
You can use unrestricted recursion to solve this problem.
\end{verbatim}
\end{question}

%%%%%%%%%%%%%%%%%%%%%%%%%%%%%%%%%%%%%%%%%%%%%%%%%%%%%%%%%%%%%%%%%%%%%%%%
\vfill\newpage
%%%%%%%%%%%%%%%%%%%%%%%%%%%%%%%%%%%%%%%%%%%%%%%%%%%%%%%%%%%%%%%%%%%%%%%%

\end{document}
